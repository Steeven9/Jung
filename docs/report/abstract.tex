\begin{abstract}

    \addchaptertocentry{\abstractname}

    % performance analysis
    % Freud
    % Jung
    % Preview of salient results

    Engineering software systems for performance is a particularly
    complex task.  The performance of a software systems depends on
    many factors, such as the algorithmic nature of the software, the
    features of the execution environment, including memory, storage,
    and CPU, and the complex interactions between components, whether
    internal or external.  A common way to support performance
    analysis is to create models of the system.  This can be done, for
    example, with traditional profilers, which characterize the
    distribution of the CPU time expenditures on functions and
    methods.

    Freud is a software performance analysis tool that derives similar
    but more expressive models called performance annotations.  In
    essence, a performance annotation characterize a relevant
    performance metric (e.g., CPU time) as a function of one or more
    input parameters or features.  In particular, Freud derives
    performance annotations automatically based on measurements of a
    software system.  The goal of this project, called \emph{Jung}, is
    to extend Freud to instrument and collect data from
    \emph{distributed} systems. This means augmenting the existing
    implementation to instrument and collect performance metrics from
    a number of distributed components.  Jung applies this
    instrumentation and collects data through a communication
    framework, and subsequently links and merges independent component
    traces using causal relations.  The resulting unified trace is
    then fed into the Freud statistical analyzer to produce meaningful
    annotations.

    Jung can therefore be used to support the performance engineering
    of applications such as those that use and connect to external
    resources, such as databases and other services.

\end{abstract}
